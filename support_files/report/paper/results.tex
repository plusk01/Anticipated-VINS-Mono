% !TEX root=../main.tex

\section{Results}\label{sec:results}

To test our implementation, we used the EuRoC datasets~\cite{Burri2016}. We selected an easy case and hard case of the EuRoC datasets to test the algorithm, MH\_02 and MH\_04. Timing data for the new modules in Anticipated VINS-Mono is given in Table~\ref{tab:timing} Running this algorithm against VINS-Mono without anticipated feature selection would not be a good comparison since VINS-Mono can select up to 150 features and would be expected to do better than any method selecting fewer features. In order to have a benchmark to test the attention algorithm against, we made two test cases: \textit{quality} and \textit{random}. In \textit{quality}, we run VINS-Mono without the anticipation algorithm, but limit the maximum features it can select to the size of $\kappa$ we use for the attention algorithm. For \textit{random}, we conduct feature selection randomly, choosing at random from the set of new features seen to satisfy the same $\kappa$ constraint. The results are given in Figure~\ref{tab:results}. \\
We can see that Anticipated VINS-Mono does better than both \textit{quality} and \textit{random} in certain cases. With the 10 features on the easier dataset, Anticipated VINS-Mono does better than \textit{random} and better than \textit{quality} as quality ended up being unstable. With 30 features, Anticipated VINS-Mono does better than \textit{quality} and \textit{random} again, with lower absolute translational error (ATE) and rotational translational error (RTE). However, Anticipated-VINS Mono does not perform well on MH\_05, the more difficult dataset, whereas \textit{quality} still produces a meaningful pose estimate.
\\
There are a few reasons that could explain why Anticipated VINS-Mono does not work perfectly. Calculating $\Omega_k^{prior}$ was not implemented in the algorithm itself due to time constraints in retrieving it from VINS-Mono backend. Including $p_l$ was also not included due to time constraints. At times, the processed poses showed that the pose (and error) was unstable and kept increasing (e.g. in Anticipated VINS-Mono run on MH\_05 with 30 features). Moreover, we did not look at the VINS-Mono backend optimizer, and it is possible that Anticipated VINS-Mono could be improved there.

\begin{figure}
	\centering
	\includegraphics[width=\columnwidth]{tableVNAVresults.png} 
	\caption{Results of Anticipated VINS-Mono on EuRoC datasets against benchmarks \textit{quality} and \textit{random}}
	\label{tab:results} 
\end{figure}

\begin{table}[h]
    \centering
    \caption{Timing Statistics on EuRoC MH\_05\_difficult}
    \begin{tabular}{clc}
        \toprule
        Thread & Component & Time (ms) \\
        \midrule
        1 & Feature Tracker & 18 \\
        \midrule
        2 & Feature Selector & 9 \\
          & Windowed Optimization (1 sec) & 30 \\
        \bottomrule
    \end{tabular}
    \label{tab:timing}
\end{table}
